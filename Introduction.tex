\chapter{Introduction}
%%%Main goals and define all the terms in the thesis title
%%%set the scene for the project report, including aims and objectives, introduce stakeholders and topic area provide report overview

\section{Motivation}
Prices for goods and services are ever-changing, constantly affecting individuals, organizations, and governments.
These clients would all benefit from the ability to predict the rise and fall of prices as it can impact their choices to spend, invest, or save.
Ultimately allowing them to make the most out of the resources they have.
\par
However, the value of inflation is an incredibly difficult indicator to predict with even establishments like the Bank of England, who have direct control over interest rates, failing to correctly forecast it.
With the ability to accurately predict inflation having so many stakeholders and huge organizations working on the problem, I am under no illusion that an undergraduate like myself will be able to find a solution better than all those before me.
Instead, I aim to understand which machine learning (ML) methods produce the best results when faced with this task and potentially find out why this is the case.
This project's results will ultimately contribute to the ongoing research into inflation forecasting and help clarify which ML methods may be most suitable. 
\section{Aims and Objectives}
\subsection{Aims}
This project aims to create multiple machine learning models and compare their predictive abilities when it comes to UK inflation.
The findings will then be presented in this report, outlining the advantages and disadvantages of each tested model.
A conclusion will then be made about which models performed best and in which scenarios certain models should be used over others.
\subsection{Objectives}
This project can be broken down into a list of objectives that not only provide a strong path to follow to complete the project but also a way to evaluate the project's success post-completion.
I have placed these objectives into phases corresponding to the project's work plan. 
\begin{enumerate}
    \item Phase 1: Research
    \begin{enumerate}
        \item Conduct a literature review to understand the current landscape of inflation forecasting.
        \item Research prominent models in the literature.
    \end{enumerate}
    \item Phase 2: Source Data
    \begin{enumerate}
        \item Source the data to be used in the models.
        \item Evaluate the usefulness and appropriateness of the data.
        \item Clean the data: dropping and retaining specific variables.
    \end{enumerate}
    \item Phase 3: Creating Models
    \begin{enumerate}
        \item Choose at least 3 appropriate models to use.
        \item Develop and tune the models.
        \item Train the models on the dataset. 
    \end{enumerate}
    \item Phase 4: Evaluation and Report
    \begin{enumerate}
        \item Evaluate the models using statistical tests.
        \item Conclude the findings.
        \item Present the findings in this final report.
    \end{enumerate}
\end{enumerate}


%%%optional...
%%%\section{Potential Risks and Constraints}
%%%\section{Methodology}