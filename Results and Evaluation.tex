\chapter{Results and Evaluation}
This chapter covers the overall evaluation of the products developed.
This includes a comparison and evaluation of the results produced by the models as well as a reflection of what could be done differently if the project were to start over.

\section{The Models'Results}
\subsubsection{Linear Regression Model, Random Forest Regression Model, and Support Vector Regression Model Predictions}
The table of regression metrics in the previous chapter showed how the linear regression, random forest, and support vector regression algorithms all performed worse prediction methods than the mean of the data (as shown in the r-squared score).
Although this was a disappointing result, it was somewhat expected as inflation is an extremely complex variable and simpler algorithms will likely struggle to form an understanding of its properties.
\begin{figure}[H]
    \centering
    \includegraphics[width=0.75\textwidth]{threemodelsres.png}
    \caption[A Comparison of the Predictions of Linear Regression, Random Forest Regression, and Support Vector Regression Models.]{A Comparison of the Predictions of Linear Regression, Random Forest Regression, and Support Vector Regression Models.}
    \label{fig:threemodelsres}
\end{figure}
Figure \ref{fig:threemodelsres} shows the predictions of each model compared to the actual normalised value (shown in black).
It can be seen that all three models struggle to accurately predict inflation.
In particular, around the 65th-70th month mark, all three models predict an increase in inflation where there is a decrease.
This seems to be due to a linear property indicating that inflation was likely to increase, however, possibly due to factors outside of the feature set, the increase was held off.

\subsubsection{LSTM Network Model Predictions}
According to the metric testing, the LSTM's results were an improvement upon the results of the previous three models, however, it still struggled to accurately predict inflation.
This is seen further in the graph comparing the LSTM's predictions to the actual values.
The predictions fail to predict the more erratic highs and lows of inflation, instead producing a much smoother curve.
This could potentially be due to the number of past values the LSTM is given.
As it stores 12 past values it may struggle to predict the sharp gains and losses of inflation.
\begin{figure}[H]
    \centering
    \includegraphics[width=0.85\textwidth]{lstmtotal.png}
    \caption[Predictions from the LSTM Network Trained on the Entire Dataset.]{LSTM Network Trained on the Entire Dataset.}
    \label{fig:lstmtotal}
\end{figure}
\begin{figure}[H]
    \centering
    \includegraphics[width=0.85\textwidth]{lstm150.png}
    \caption[Predictions from the LSTM Network Trained on a subset of the Dataset.]{LSTM Network Trained on a subset of the Dataset.}
    \label{fig:lstm150}
\end{figure} 
The graph produced by the LSTM that was trained on the subset of the dataset is preferable to the graph produced by the LSTM trained on the entire dataset.
However, neither model produces close accurate predictions to the the actual value of inflation.
Both LSTMs can follow the trends of inflation (although somewhat delayed) but struggle to produce predictions that are of the correct magnitude.

%%%%%%%%%%%%%%%%%%%%%%%%%%%%%%%
% Include figure for LSTM
% Further analysis if needed
% Talk about the training data as well
% over trained or not
%%%%%%%%%%%%%%%%%%%%%%%%%%%%%%%
\subsubsection{Aritficial Neural Network Predictions}
Of the models implemented, the custom ANN produced the best results.
This model, like the others, was implemented on a subset of the original data that did not include values from before the 1980s.
\begin{figure}[H]
    \centering
    \includegraphics[width=0.85\textwidth]{anntotal.png}
    \caption[Predictions from the Artificial Neural-Network Trained on the Entire Dataset.]{Neural-Network Trained on the Entire Dataset.}
    \label{fig:anntotal}
\end{figure}
\begin{figure}[H]
    \centering
    \includegraphics[width=0.85\textwidth]{ann150.png}
    \caption[Predictions from the Artificial Neural-Network Trained on a subset of the Dataset.]{Neural-Network Trained on a subset of the Dataset.}
    \label{fig:ann150}
\end{figure} 

Figures \ref{fig:anntotal} and \ref{fig:ann150} show the results produced by the model with the entire data vs the subset of the data respectively.
The figures show how the models trained on the subset of the dataset produced more accurate predictions than the models trained on the entire dataset.

\subsubsection{Training and Testing the Models on the Entire Data vs on a Subset}
% %TODO
% - add the lstm figuers
% - add the trainingfigures
% - talk about overfitting/training, how many epochs used and how we know it wasnt overtrianed.
% - talk about the whole set fs the set without hte first 150 points
% - compare the results of each model
% - which was best, which was worst
% - is this the same as was predicted?
% - why did the good ones do good and the bad do bad
% - what could be improved next time?
% - why does random forest do bad? Each tree is trained on a RANDOM subset of the data (not good for sequential)
% - why does SVR do bad? Trys to fit a hyperplane of best fit, this might not be possible if the datas relationships has more then 3 dimensions
% - why does linear regression do bad? there are no linear relationships


BIG TODOS
- INFLATION DATA SHOULD BE SHIFTED (YOU CANT PREDICT THE INFLATION OF THIS YEAR WITH THIS YEARS ECONOMIC DATA)
- INFLATION SHOULD NOT BE INCLUDED IN THE TESTING SET UNLESS SHIFTED
- Implement and write about shift distance!