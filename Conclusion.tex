\chapter{Conclusion}
With inflation's large influence on society and the constant research conducted within machine learning; knowing which models are the most suitable for forecasting inflation and similar macroeconomic problems offers a compelling foundation for future research to compound upon.
This project provides such a foundation for future endeavors by proposing the use of feedforward neural networks and long short-term memory networks for complex time series forecasting whilst ruling out the use of simpler models such as linear regression, random forest regression, and support vector regression.
Through implementation and analysis, the positives and negatives of each model have been discussed in this report as well as their suitability to the task of predicting inflation.

\section{Evaluation Objectives}
The original aim of this project was to create several machine learning models and compare their ability to forecast inflation, ultimately concluding which models were best suited for the task and which models were not.
This aim has been fulfilled with the findings that the feedforward neural network (FNN) is currently the best suited to predicting inflation due to its strong performance despite the limited dataset.
In this project, the LSTM performed worse than the FNN but LSTMs may still have the potential to create better predictions given a large enough dataset.
Unfortunately, as inflation is a macroeconomic factor that is calculated a limited number of times a year, a dataset of this magnitude would prove difficult to produce.
As predicted, the linear regression, random forest, and support vector regression models struggled to deal with the complexity of inflation and its relationships.
These three models are unlikely to be favourable choices when attempting to forecast inflation or other complex financial time series.

The objectives of this project outlined the development process and the goals that need to be accomplished throughout it.
Each of these objectives has successfully been completed and documented throughout this report.

\subsection{Process Evaluation}
The choice of environment (Jupyter) and programming language (Python) posed no issues and provided plenty of functionality that aided the work process such as Python's extensive libraries and being able to run individual cells in Jupyter.
Moreover, each of the libraries that were used had comprehensive documentation as well as active communities that helped to provide answers to many of the problems encountered during programming.

By utilising the machine learning project flow proposed by Chip Huyen\cite{Huyen_2022} the design and implementation portions of the project were completed efficiently.
Additionally, the book "Projects in Computing and Information Systems: A Student's Guide" by Christian Dawson\cite{dawson2005projects} was a tremendous help throughout the entire report-writing process.
Thanks to these two texts and several others alongside the guidance of my supervisor, Mohamad Saada, the process of researching, designing, and implementing this project was able to run smoothly.

\section{Shortcomings and Future Improvements}
Throughout this project, numerous issues had to be overcome and several improvements could have been made.
'Predicting Inflation with Machine Learning' required a strong foundational knowledge of machine learning that had to be built and developed from start to finish of the project.
This required nonstop study of the current literature and countless hours spent reading documentation and community forums in order to ensure the implementations of each model were correct and of an acceptable standard.
As the project progressed this knowledge began to compound and tasks that would have been laborious at the start of the year became intuitive by the final stages.

The project analysed the results of five different machine learning models tasked with forecasting inflation.
Given more time and resources, the results from these models could be improved upon by using a larger dataset or by further tuning and experimentation with the hyperparameters and architectures of each model.
Training time for the feedforward and LSTM networks proved to be somewhat of a bottleneck which could have been reduced by using computers with greater processing power.

\section{Looking Forwards}
This project's results demonstrate machine learning's potential when applied to a broader economic context and its ability to forecast complex financial time series such as inflation.
These results suggest that, in the future, more time should be dedicated to developing deeper feedforward neural networks aimed at forecasting inflation with even greater accuracy rather than the other models explored.
Findings in this report can potentially be applied to other areas of macroeconomics such as developing LSTMs for forecasting indicators that have large amounts of available data like the S\&P-500, 
or utilising feedforward networks for forecasting economic indicators that do not have long-term temporal relationships or large enough datasets to exhibit these relationships.

 
\section{Final Thoughts and Reflections}
Looking back on this project, I believe that the initial objectives were met and the title 'Predicting Inflation with Machine Learning' was fulfilled.
Although not every model created in the process of this project produced amazing results (or even good results for that matter) I learned something new from the implementation and analysis of each one.
I am proud of the work I have managed to produce over the past year of study and I think the skills I have picked up through this work will will no doubt aid me for years to come.
I hope that the results found through the completion of this report will encourage further research in this field as inflation is something that strongly impacts everyone's lives and improving our understanding of it can only be a net positive for everyone.

% TODO:
% Conclusion:
% - how does the projetc fit into/enhance existing work in the field
% - waht problems were faced? how were they overcome?
% - what would you do if u had more time/ next time
% - what would you do differently
% - what have you learnt and experienced
% - how do you recommend the project should be taken forward.
% Evaluation:
% - was the development process appropriate
% - was the programming language suitable
% Overall:
% - having done this work we are now in a position to do X
% - explain the limiations and applications of the work