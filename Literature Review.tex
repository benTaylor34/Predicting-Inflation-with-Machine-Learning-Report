\chapter{Literature Review}

\section{Motivation}%%%why lit review? what does it do?
Embarking on a literature review before developing our project offers numerous benefits. 
Understanding existing knowledge in Machine Learning, specifically when used to predict financial indicators, 
helps to contextualise our research, positioning it within the existing field. 
Reviewing previous literature also provides the benefits of identifying gaps in current research and
finding supporting arguments that can guide our work and help us to avoid, as much as possible, 
redundancy in our and others' works. 
Having completed the literature review, we should have a strong foundation to start and guide our project.

\section{Available Literature}
This report's topic focuses on the prediction of inflation through the use of machine learning.
To accomplish this we can view papers predominantly addressing two types of topics. 
The first type is papers that focus on the topic of predicting inflation or other economic indicators and time series.
The second type of papers we can research are ones that deal with different machine learning techniques.
Additionally, it is pertinent to survey the current literature on inflation: its causing factors, effects, and significance.  
\subsection{Financial Indicator Prediction Papers}
The application of using past values to predict future values has been widely implemented for years, 
one of the earliest uses of autoregressive models used for time series being in the 1920s from U.G.Yule\cite{5c7b6f25-ed11-3745-8118-935d66a8f3d3}.
There is a strong monetary incentive to produce research on how best to predict financial indicators as the correct predictions can lead to profit and the avoidance of loss both for corporations and for individuals. 
This results in a large assortment of papers written experimenting with a variety of techniques to predict future values, most of which we can learn from to help structure our models.
The predominant analysis strategies for predicting financial market behaviour are fundamental analysis and technical analysis\cite{Harrington_2003}.
Fundamental analysis attempts to measure the intrinsic value of an asset by looking at current market and economic conditions.
Technical analysis attempts to identify opportunities and predict investments by viewing movements and trends in market data.
\subsection{Machine Learning Papers} 
Machine Learning is a 'hot topic' that is to say there is an abundance of fresh papers constantly being put out within the field. 
This bodes well for our project as it means that we should have plenty of guidance on the options available to conduct and develop our predictive models. 
\subsection{Inflation Papers}

\section{Problem Domain} 
\subsection{Inflation a continuation of "Inflation Papers"}

\subsection{Existing Models}
According to "Analysis of Financial Time Series" by Ruey S. Tsay "Financial time series analysis concerned with the theory and practice fo asset valuation over time."\cite{tsay2005analysis}
There are many financial time series (FTS for short) prediction methods both theoretical and practical that have attracted attention. 
Among these is the trading discipline of technical analysis\cite{achelis2001technical}. 
Technical analysis uses a variety of technical indicators and patterns in market data to evaluate and make predictions on investments and FTS.
There are three main categories of technical analysis which are statistical models, Machine Learning (ML) models, and hybrid models. 
Statistical models 
Machine learning models
Hybrid models 
\subsubsection{None Machine Learning Models}
\subsubsection{Machine Learning Models}
\subsection{Gap Anaylsis}
\subsubsection{Financial Markets and The Current State of Inflation}
\subsubsection{Machine Learning vs Traditional Methods for Financial Forecasting}
\subsubsection{Feature Comparison}


\section{Summary and Conclusion}