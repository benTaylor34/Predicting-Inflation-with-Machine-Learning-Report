\chapter{Literature Review}

\section{Motivation}%%%why lit review? what does it do?
Embarking on a literature review before developing our project offers numerous benefits. 
Understanding existing knowledge in Machine Learning, specifically when used to predict financial indicators, 
helps to contextualise our research, positioning it within the existing field. 
Reviewing previous literature also provides the benefits of identifying gaps in current research and
finding supporting arguments that can guide our work and help us to avoid, as much as possible, 
redundancy in our and others' works. 
Having completed the literature review, we should have a strong foundation to start and guide our project.

\section{Available Literature and Context}
There is certainly a strong monetary incentive to produce research on how best to predict financial indicators. 
The correct predictions can not only allow organizations and individuals to profit greatly but also to avoid loss.
This results in a myriad of papers being written, experimenting with a variety of techniques to predict future values, 
most of which we can learn from to help structure our models.
\par 
This report's topic focuses on the prediction of inflation through the use of machine learning.
To accomplish this we can view papers predominantly addressing two types of topics. 
The first type is papers that focus on the topic of predicting inflation or other economic indicators and time series.
The second type of papers we can research are ones that deal with different machine learning techniques.
Additionally, it is pertinent to survey the current literature on inflation: its causing factors, effects, and significance.  

\subsection{Financial Indicator Prediction Papers}
According to "Analysis of Financial Time Series" by Ruey S. Tsay 
"Financial time series analysis is concerned with the theory and practice of asset valuation over time."\cite{tsay2005analysis}
There are many financial time series (FTS for short) prediction methods both theoretical and practical that have attracted attention. 
The predominant analysis strategies for predicting financial market behaviour are fundamental analysis and technical analysis\cite{Harrington_2003}.
\subsubsection{Fundamental Analysis} 
Fundamental analysis\cite{thomsett2006getting} attempts to measure the intrinsic value of an asset by looking at current market and economic conditions.
Additionally, fundamental analysis frequently makes use of techniques - such as sentiment analysis - that often deal with unstructured data.
The success of fundamental analysis often relies on the financial efficiency of the target\cite{WAFI2015939},
which according to Tom Seegmiller is defined as
"... how successful your organization is at turning expenses into revenue"\cite{Seegmiller_2023}.
\subsubsection{Technical Analysis}
Technical analysis\cite{achelis2001technical} attempts to identify opportunities and predict investments by viewing movements and trends in market data alongside using a variety of technical indicators.
Unlike fundamental analysis, technical analysis does not take into account many of the same fundamentals that can help indicate an asset's current value such as quarterly revenue.
This is partially because it is often argued that technical indicators such as inflation or a stock's value are already priced according to the fundamentals that cause or contribute to them\cite{murphy1999technical}.
From this, we can come to the understanding that while fundamental analysis is the idea of looking at the current factors affecting an asset and using them to evaluate to asset's true value; 
Technical analysis is built upon the idea that past performance can predict future performance.
Traditionally, technical analysis has relied heavily on statistical models to forecast the future performance of assets\cite{rockefeller2019technical}.
Furthermore, the application of using past values to predict future values has been widely implemented for years, 
with one of the earliest uses of autoregressive models being used to predict time series created by U.G.Yule in the 1920s\cite{5c7b6f25-ed11-3745-8118-935d66a8f3d3}.
However, with the increase of big data and the internet, ever-larger amounts of financial predictive data are continually being produced.
Nowadays, simple statistical models may struggle to produce accurate future predictions when faced with big data sets containing complex characteristics\cite{Akbilgic2014ANH}.

\subsection{Machine Learning Papers} 
This brings us to machine learning algorithms\cite{michalski2013machine}. 
According to Mariette Awad et al. machine learning "is a branch of artificial intelligence that systematically applies algorithms to synthesize the underlying relationships among data and information"\cite{Awad2015}.
Currently, there is a massive abundance of fresh machine learning papers constantly being produced in the field. \cite{8259424}
\begin{figure}[H]
    \centering
    \includegraphics[width=0.75\textwidth]{articleGrowth.png}
    \caption[ML arXiv articles per year]{ML arXiv articles per year. \par \textit{Figure from page 4 of 'A New Golden Age in Computer Architecture: Empowering the MachineLearning Revolution' by Jeff Dean, David Patterson, and Cliff Young}\cite{8259424}}
    \label{fig:articleGrowth}
\end{figure}
As you can see in figure \ref{fig:articleGrowth} articles on machine learning posted to arXiV (an archive for scholarly articles) have more than doubled every two years.
Additionally, in 2018 the number of articles released reached 100 per day, summing to more than 33,000 by the end of the year.
The number of articles released has steadily continued to increase in the years since\cite{Dean2019TheDL}.
Naturally, to read this many articles is impossible, 
however, the sheer quantity bodes well for this project as it means there will be plenty of guidance on how best to select and develop our predictive models. 
\par By utilising machine learning techniques in financial forecasting we can endeavor to improve upon the performance of traditional statistical models.

\subsection{Inflation Papers}

\section{Problem Domain} 
\subsection{Inflation a continuation of "Inflation Papers"}
\subsection{Existing Models}
\subsection{Machine Learning Models}
\paragraph{Artificial Neural Networks}
\paragraph{Support Vector Regression}
\paragraph{Random Forest}
\subsection{Gap Anaylsis}
\subsubsection{Financial Markets and The Current State of Inflation}
\subsubsection{Machine Learning vs Traditional Methods for Financial Forecasting}
\subsubsection{Feature Comparison}


\section{Summary and Conclusion}