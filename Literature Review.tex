\chapter{Literature Review}

\section{Motivation}
Embarking on a literature review before developing our project offers numerous benefits. 
Understanding existing knowledge in Machine Learning, specifically when used to predict financial indicators, 
helps us contextualise our research positioning it within the existing field. 
Reviewing previous literature also gives us the benefits of identifying gaps in current research; 
finding supporting arguments that can help guide our work and helping us to avoid, as much as possible, 
redundancy produced by our efforts. 
Having completed the literature review, we should have a strong foundation to start and complete our project.

\section{Available Literature}
Machine Learning is a 'hot topic' that is to say there is an abundance of fresh papers constantly being put out within the field. 
This bodes well for our project as it means that we should have plenty of guidance on the options available to conduct and develop our predictive models. 
\par Additionally, there is plenty of monetary incentive to produce research on how best to predict various financial time series. 
This results in a variety of papers written with a variety of techniques used, most of which we can learn from to help structure our model.

\section{Problem Domain}

\section{Problem Solution}