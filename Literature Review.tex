\chapter{Literature Review}

\section{Motivation}
Embarking on a literature review before developing our project offers numerous benefits. 
Understanding existing knowledge in Machine Learning, specifically when used to predict financial indicators, 
helps to contextualise our research, positioning it within the existing field. 
Reviewing previous literature also gives the benefits of identifying gaps in current research; 
finding supporting arguments that can guide our work and help us to avoid, as much as possible, 
redundancy produced by our efforts. 
Having completed the literature review, we should have a strong foundation to start and guide our project.

\section{Available Literature}
Machine Learning is a 'hot topic' that is to say there is an abundance of fresh papers constantly being put out within the field. 
This bodes well for our project as it means that we should have plenty of guidance on the options available to conduct and develop our predictive models. 
\par Additionally, there is plenty of monetary incentive to produce research on how best to predict various financial time series. 
This results in a variety of papers written with a variety of techniques used, most of which we can learn from to help structure our model.
 
\section{Problem Domain}
Due to the monetary gain to be made from trading and predicting financial time series (FTS), the literature and articles written are numerous. 
There are many FTS prediction methods both theoretical and practical that have attracted attention. 
Among these is the trading discipline of technical analysis. 
Technical analysis uses a variety of technical indicators and patterns in market data to evaluate and make predictions on investments and FTS.
There are three main categories of technical analysis which are statistical models, Machine Learning (ML) models, and hybrid models. 
Statistical models 
Machine learning models
Hybrid models 

\section{Problem Solution}